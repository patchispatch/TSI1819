\documentclass[11pt,a4paper]{article}
\usepackage[utf8]{inputenc}
\usepackage[spanish]{babel}	%Idioma
\usepackage{amsmath}
\usepackage{amsfonts}
\usepackage{amssymb}
\usepackage{graphicx} %Añadir imágenes
\usepackage{geometry}	%Ajustar márgenes
\usepackage[export]{adjustbox}[2011/08/13]
\usepackage{float}
\restylefloat{table}
\usepackage{hyperref}
\usepackage{titling}
\usepackage{minted}
\usepackage{multicol}
%Opciones de encabezado y pie de página:
\usepackage{fancyhdr}
\pagestyle{fancy}
\lhead{}
%\rhead{}
\lfoot{Técnicas de los Sistemas Inteligentes}
\cfoot{}
\rfoot{\thepage}
\renewcommand{\headrulewidth}{0.4pt}
\renewcommand{\footrulewidth}{0.4pt}
%Opciones de fuente:
\renewcommand{\familydefault}{\sfdefault}
\setlength{\parindent}{0pt}
\setlength{\headheight}{12pt}
\setlength{\voffset}{10mm}

% Custom colors
\usepackage{color}
\definecolor{deepblue}{rgb}{0,0,0.5}
\definecolor{deepred}{rgb}{0.6,0,0}
\definecolor{deepgreen}{rgb}{0,0.5,0}

\usepackage{listings}

\pretitle{%
  \centering
  \LARGE
  \includegraphics[scale=0.5]{img/logo.png}\\[\bigskipamount]
}
\posttitle{\begin{center} \end{center}}

\author{	Juan Ocaña Valenzuela}
\title{\textbf{Práctica 2: planificación clásica en PDDL}}

\begin{document}

\thispagestyle{empty}

\maketitle

\newpage

\tableofcontents

\newpage

\section{Diario de trabajo}

\subsection*{Domingo, 5 de mayo de 2019}
Empecé la práctica hace unos días, pero no ha sido hasta hoy que he decidido comenzar a documentarla.
Llevaba unos días bastante ocupado (la fiesta de la Escuela no se organiza sola), y tenía visita en casa, así que dedicaba
relativamente poco espacio de mi sistema nervioso central a pensar en los apasionantes mundos de Belkan, más allá de la 
eterna duda: ¿es Belkan el personaje? ¿es el nombre del mundo? ¿o quizá un ser omnipresente que gobierna por aquellos lares?


Una vez mi visita se marchó, comencé a plantear el primer ejercicio de la relación, diseñando un mapa, poniendo cosas en él,
definiendo las acciones y todo eso. Una vez creí haber terminado el modesto problema 1, en el cual cada personaje debe tener un
objeto, ejecuté FF y...


\textbf{no funcionaba.}

\bigskip

No encontraba un plan. No lo encontraba y no sabía por qué. Estaba cansado y no me apetecía hacer nada, así que mandé a la
mierda los estrafalarios mundos de Belkan y me puse a jugar a Persona 5.

\medskip

Hoy es un día nuevo, y nada más levantarme y desayunar he pensado que quizás podría hacer un pequeño esfuerzo por saber 
qué le pasaba a mi problema, y cómo arreglarlo. Y es que nuestro pequeño bicho no puede entregar objetos a los personajes
si no lo hemos definido en el problema. Qué cosas, eh.

\medskip

Bautizado como Patrick Dotimas (Patrick para los amigos), ahora que está en el problema encuentra una solución.
Ya tenemos un problema resuelto... O no.

\medskip

Observando el plan, resulta que, en un acto de amor incondicional, Patrick se entrega a sí mismo a los personajes, y no los
objetos que coge. En lugar de algo como:

\medskip
\begin{center}
\texttt{GIVE PATRICK ROSE R20 PRINCESS}
\end{center}

\medskip

tenemos algo así:

\medskip
\begin{center}
\texttt{GIVE PATRICK PATRICK R20 PRINCESS}
\end{center}

\medskip

A ver cómo lo soluciono... De momento voy a comer.

\bigskip

\textbf{[Actualización - 19:30]}

Después de procrastinar un buen rato (y jugar otro rato a Persona 5), me he dado cuenta de que el dominio no
estaba mal del todo. Patrick cogía sus objetos y los entregaba bien, pero no sabía qué objeto tenía ni daba.
Lo he solucionado añadiendo un predicado más, \texttt{on\_hand}, que viene explicado en su sitio.

\medskip

Acabo de caer en una cosa... Si Belkan es el personaje, y nuestro jugador se llama Patrick, 
¿son estos los extraños mundos de Patrick? 
En fin, voy a ponerme con el generador de problemas, a ver si marcha. 

\medskip

\texttt{Continuará...}

\section{Leyenda}

\texttt{TO DO}.

\section{Ejercicios}

\subsection{Ejercicio 1}

Definir  un  dominio y  problema de  planificación considerando  que  el  jugador podrá    estar  orientado  al  norte,  sur,  este  u  oeste  y  desplazarse  de  una  zona  a  otra siempre  que  esté  correctamente  orientado.  Por  ejemplo,  podrá  desplazarse  a  una zona  al  norte  de  su  zona  actual,  si  está  orientado  al  norte.

\subsubsection{Apartado A}

\textbf{Representar en el dominio los objetos del mundo.}

\bigskip

Para representar los objetos del mundo (jugador, personajes, objetos, habitaciones, caminos, etc.) se han establecido
los siguientes tipos, con su jerarquía:

\medskip

\textbf{locatable:} elemento que se puede localizar en una posición.
	
\quad \textbf{character:} personaje.

\quad \quad	\textbf{player:} jugador.

\quad \quad \textbf{npc:} personaje no jugador.

\quad \textbf{object:} objeto.

\textbf{orientation:} orientación de un elemento.

\textbf{room:} habitación del dominio.

\subsubsection{Apartado B}

\textbf{Representar predicados que permitan describir los estados del mundo.}

\bigskip

Se han considerado los siguientes predicados:

\medskip

\large{\textbf{at}}

\texttt{(at ?r - room ?l - locatable)}

\smallskip

Un elemento \texttt{?l} se encuentra en la habitación \texttt{?r}. 

\medskip

\large{\textbf{on\_floor}}

\texttt{(on\_floor ?o - object)}

\smallskip

Un objeto \texttt{?o} se encuentra en el suelo. 

\medskip

\large{\textbf{compass}}

\texttt{(compass ?o - orientation)}

\smallskip

La orientación del personaje es \texttt{?o}.

\medskip

\large{\textbf{path}}

\texttt{(path ?r1 ?r2 - room ?o - orientation)}

\smallskip

Existe un camino entre \texttt{?r1} y \texttt{?r2}, en el que la segunda habitación
se encuentra con una orientación \texttt{?o} respecto de la primera.

\medskip

\large{\textbf{has\_object}}

\texttt{(has\_object ?c - character)}

\smallskip

Un personaje \texttt{?c} tiene un objeto.

\medskip

\large{\textbf{on\_hand}}

\texttt{(on\_hand ?o - object)}

\smallskip

El jugador tiene un objeto \texttt{?o} en la mano.

\subsubsection{Apartado C}

\textbf{Representar las siguientes acciones del jugador: girar a la izquierda, girar a la derecha, ir, coger, dejar, entregar.}

\bigskip

\large{\textbf{Girar a la izquierda}}

\texttt{(:action TURN\_LEFT)}

\smallskip

Dada una orientación, el jugador mirará a aquella a su izquierda:

\begin{itemize}
\item N $\rightarrow$ W
\item W $\rightarrow$ S
\item S $\rightarrow$ E
\item E $\rightarrow$ N
\end{itemize}

\medskip

\large{\textbf{Girar a la derecha}}

\texttt{(:action TURN\_RIGHT)}

\smallskip

Dada una orientación, el jugador mirará a aquella a su derecha:

\begin{itemize}
\item N $\rightarrow$ E
\item W $\rightarrow$ N
\item S $\rightarrow$ W
\item E $\rightarrow$ S
\end{itemize}

\medskip

\large{\textbf{Girar 180 grados}}

\texttt{(:action TURN\_180)}

\smallskip

Dada una orientación, el jugador mirará a aquella a su espalda:

\begin{itemize}
\item N $\rightarrow$ S
\item W $\rightarrow$ E
\item S $\rightarrow$ N
\item E $\rightarrow$ W
\end{itemize}

\medskip

\large{\textbf{Coger un objeto}}

\texttt{(:action PICK)}

\smallskip

Si el jugador se encuentra en la misma habitación que un objeto y este se halla en el suelo,
el jugador podrá cogerlo. El objeto dejará de estar en la habitación y en el suelo.

\medskip

\large{\textbf{Soltar un objeto}}

\texttt{(:action DROP)}

\smallskip

Si el jugador tiene un objeto, lo soltará. El objeto pasará a estar en la misma habitación que el jugador,
y en el suelo.

\medskip

\large{\textbf{Dar un objeto a un personaje}}

\texttt{(:action GIVE)}

\smallskip

Si el jugador tiene un objeto, el npc no, y ambos se encuentran en la misma habitación, el jugador le dará 
el objeto al npc. El jugador pasará a no tener objeto, y el npc sí lo hará. 

\medskip
\large{\textbf{Moverse a una habitación contigua}}

\texttt{(:action GO)}

\smallskip

El jugador se moverá a la habitación hacia la que esté orientado.

\medskip

\subsubsection{Apartado D}

\textbf{Plantear un problema de planificación con un estado inicial con 25 zonas conectadas arbitrariamente en el que aparezcan situados los 5  personajes en distintas zonas y al menos 5 objetos. El objetivo de este problema consistirá en conseguir que todos los personajes  tengan al menos un objeto.}

\bigskip

El mapa planteado en este problema es el siguiente:

\texttt{TO DO}

\medskip

El objetivo es que todos los personajes dispongan de un objeto, es decir, que se cumpla:

\medskip

(\texttt{AND}

\quad \texttt{(has\_object leonardo)}

\quad \texttt{(has\_object prince)}

\quad \texttt{(has\_object princess)}

\quad \texttt{(has\_object professor)}

\quad \texttt{(has\_object witch)}

)

\bigskip

El plan generado por FF es el siguiente:

\begin{multicols}{3}
0: TURN\_180 N
        1: GO PATRICK R13 R18 S
        
        2: TURN\_180 S

        
        3: PICK PATRICK APPLE R18

        4: TURN\_LEFT N

        5: GO PATRICK R18 R17 W

        6: GO PATRICK R17 R16 W

        7: TURN\_RIGHT W

        8: TURN\_RIGHT N

        9: GO PATRICK R16 R17 E

       10: TURN\_RIGHT E

       11: GO PATRICK R17 R22 S

       12: TURN\_LEFT S

       13: GIVE PATRICK APPLE R22 WITCH

       14: GO PATRICK R22 R23 E

       15: GO PATRICK R23 R24 E

       16: TURN\_LEFT E

       17: GO PATRICK R24 R19 N

       18: TURN\_RIGHT N

       19: GO PATRICK R19 R20 E

       20: TURN\_LEFT E

       21: GO PATRICK R20 R15 N

       22: TURN\_180 N

       23: PICK PATRICK GOLD R15

       24: GO PATRICK R15 R20 S

       25: TURN\_180 S

       26: TURN\_LEFT N

       27: GO PATRICK R20 R19 W

       28: TURN\_180 W

       29: GIVE PATRICK GOLD R19 PROFESSOR

       30: GO PATRICK R19 R20 E

       31: TURN\_RIGHT E

       32: GO PATRICK R20 R25 S

       33: TURN\_180 S

       34: PICK PATRICK ROSE R25

       35: GO PATRICK R25 R20 N

       36: TURN\_LEFT N

       37: GO PATRICK R20 R19 W

       38: TURN\_RIGHT W

       39: GO PATRICK R19 R14 N

       40: TURN\_LEFT N

       41: GO PATRICK R14 R13 W

       42: GO PATRICK R13 R12 W

       43: GO PATRICK R12 R11 W

       44: TURN\_RIGHT W

       45: TURN\_RIGHT N

       46: GIVE PATRICK ROSE R11 PRINCESS

       47: GO PATRICK R11 R12 E

       48: TURN\_LEFT E

       49: GO PATRICK R12 R7 N

       50: GO PATRICK R7 R2 N

       51: TURN\_LEFT N

       52: GO PATRICK R2 R1 W

       53: TURN\_LEFT W

       54: PICK PATRICK OSCAR R1

       55: GO PATRICK R1 R6 S

       56: TURN\_LEFT S

       57: GO PATRICK R6 R7 E

       58: TURN\_RIGHT E

       59: GIVE PATRICK OSCAR R7 LEONARDO

       60: GO PATRICK R7 R12 S

       61: TURN\_LEFT S

       62: GO PATRICK R12 R13 E

       63: TURN\_LEFT E

       64: GO PATRICK R13 R8 N

       65: TURN\_RIGHT N

       66: GO PATRICK R8 R9 E

       67: TURN\_LEFT E

       68: GO PATRICK R9 R4 N

       69: TURN\_LEFT N

       70: GO PATRICK R4 R3 W

       71: TURN\_180 W

       72: PICK PATRICK ALGORITHM R3

       73: GO PATRICK R3 R4 E

       74: TURN\_RIGHT E

       75: GO PATRICK R4 R9 S

       76: GIVE PATRICK ALGORITHM R9 PRINCE

\end{multicols}

\end{document}